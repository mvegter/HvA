% !TeX spellcheck = nl_NL

\competentie
{% competentieformulier
	\competentieformulier
	{% toelichting
		Je hebt kennis en vaardigheden die belangrijk zijn voor jouw rol als professional in het ICT-werkveld. Je kunt de kennis die je hebt opgedaan beoordelen op relevantie. Op basis daarvan maak je keuzes voor het toepassen ervan bij het uitvoeren en oplossen van praktijkvraagstukken. Je hanteert daarbij een methodische werkwijze, stelt criteria op waaraan het resultaat moet voldoen en werkt volgens professionele (internationale) ICT-standaarden. Je hebt een ondernemende houding.
	}
	{% deelcompetenties
		planmatig werken,%
		toepassing van (wetenschappelijke) kennis en inzichten,%
		kwaliteit leveren,%
		ondernemen%
	}
	{% beroepstaken
		1...,%
		2...,%
		Etc.%
	}
	{%
		Bewijs uit stage
	}
	{%
		Voor deze competentie mag je maximaal drie beroepsproducten opnemen als bewijs. Je mag één beroepsproduct vervangen door een ervaringsverslag. De beroepsproducten worden voorafgegaan door een toelichting in de vorm van een STARR-formulier. Een eventueel ervaringsverslag maak je ook door het STARR-formulier in te vullen.
	}
	{% verwijzing naar bewijs
		Bewijs 1,%
		Bewijs 2,%
		Bewijs 3%
	}
}
{% bewijzen
	\bewijs
	{% naam
		a
	}
	{% starr
		\starr
		{% betreft
			b
		}
		{% datum
			c
		}
		{% situatie
			d
		}
		{% taak
			e
		}
		{% activiteiten
			f
		}
		{% resultaat
			g
		}
		{% reflectie
			h
		}
	}
	{% bewijs
		
	},
	\bewijs
	{% naam
		a
	}
	{% starr
		\starr
		{% betreft
			b
		}
		{% datum
			c
		}
		{% situatie
			d
		}
		{% taak
			e
		}
		{% activiteiten
			f
		}
		{% resultaat
			g
		}
		{% reflectie
			h
		}
	}
	{% bewijs
		
	},
	\bewijs
	{% naam
		a
	}
	{% starr
		\starr
		{% betreft
			b
		}
		{% datum
			c
		}
		{% situatie
			d
		}
		{% taak
			e
		}
		{% activiteiten
			f
		}
		{% resultaat
			g
		}
		{% reflectie
			h
		}
	}
	{% bewijs
		
	}
}
